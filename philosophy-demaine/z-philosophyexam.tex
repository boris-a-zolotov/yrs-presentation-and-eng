=== European Science in the Middle Ages ===

A single series of medieval events that affected science the most is the emergence of universities. XII–XIII centuries saw the universities in Cambridge, Naples, Paris, Oxford etc. being founded. Universities would then become the centers of the scientific knowledge and studies.

The cornerstone of the medieval perception of the world was that the world was created by the God from nothing. Therefore everything exists because of the God's efforts, and anything is valuable solely because it is a part of the Divine creation. This yields the fact we can not but acknowledge: theology was the single most influential science with the most impact on other scientific domains.

Patristics was the period in theology that laid the philosophical fundament to Christianity itself, (even though a modern reader of Saint Augustine might think that he almost disproved the existence of God). St. Augustine's works are to these days the most important piece of literature employed by the church in its theoretical description of itself. St. Augustine also discussed physical matters like time and the principles it follows and criticized his own studies of antique eloquence.

Scholastics was a medieval method of learning involving the interaction between different points of view and several people defending them in a form of a straightforward dialogue. This framework was therefore easily extendable to other fields of knowledge besides theology, meaning of the God's words and descriptions of mystical objects.

One possible approach to science in the medieval circumstances is trying to find and describe the “design principles” followed by the God regarding any object or event in the world. A scientist pursuing this should then have brilliant knowledge of the Holy Bible to recognize what is pointed on there in any existing thing. However, laws of mathematics (for instance) should also have been known, since the God had for certain acted with accordance to them while creating the world. This is how the science was carried out in Paris University for example.

Another approach can though be utilized: antique findings can be considered in an attempt to conform them to the Christian perspective. This way still implies the knowledge of the Holy Bible, yet much attention should be paid to studying the nature itself. This is why astronomy, mathematics and physics have seen some great development during medieval time (in Oxford, for example). It is almost agreed upon that, despite apparent commitment to embracing the God only, medieval science in fact developed methods of research and cognition very close to those of the modern one.

=== Naturalism in Philosophy of Mathematics: its Ontological and Epistemological Consequences ===

The idea around which the concept of naturalism is revolving is that all valid properties, all reasonable objects in science and all methods of how they could be researched are in some way natural. That is, everything falls in reach of science, and there are no supernatural unseen forces that can influence the object of study: if any influence is made, it can be detected and measured. Several philosophers argue that being natural equals having a comprehensive mathematical model giving, among other things, a quantitative description to the object.

Basic naturalism does not regulate the categories employed by a scientist and the specific scientific theories used to describe the world (which remains natural and thus completely describable). Naturalism is just about refuting the explanations and proofs that appeal to non-natural forces and factors: ones that can not in turn be studied and explained further in the same way the ongoing reasoning has been done (imagine a mathematical proof referring to a little bird placing prime numbers). This somewhat relaxed regulation allows for still a wide variety of approaches to science among naturalists. 

Naturalist views in mathematics can be categorized based on what they apply to: ontological naturalism considers the notions, objects and constructions; epistemiological naturalism is concerned with the methods of interacting with them; methodological naturalism presents the standards according to which the field of mathematics should be developed.

It can be noted that to completely translate any category into a naturalistic realm, one has to to so ontologically and epistemiologically: not only instances of the category have to be specified and described in a natural way, but also methods of reasoning and collecting knowledge about them have to be made consistent and not referring to anything out of the category.

Let us consider mathematics from a naturalistic perspective. Mathematics are consistent and no mathematical statements rely on anything but other mathematical statements in an understandable logical way. What is more, mathematics to date has been aligned with the natural phenomena that one could try to describe mathematically. For a naturalist, though, this is not enough to state the naturalness of mathematics; they would want the scientific confirmation that the way the knowledge is obtained in mathematics matches the way it is done in other fields that are already proved to be naturalistic.

A semi-naturalistic approach can be demonstrated regarding mathematics: one might want to keep believing mathematical objects are purely abstract and could never occupy a point in space or time, therefore they can not be thought of as natural to any extent; however, the methods of studying those objects should be borrowed from definitely naturalistic field and should not rely on an intuition.








