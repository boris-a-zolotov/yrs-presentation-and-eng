\documentclass[a4paper,12pt]{article}

\usepackage{graphicx,amssymb,amsmath,indentfirst,hyperref}

\usepackage[margin=2.25cm]{geometry}
\parindent=0.5in \parskip=3.5mm \linespread{1.26}
\footskip=7mm

\usepackage[utf8x]{inputenc}
\usepackage[english]{babel}

\begin{document}

\thispagestyle{empty}

\newgeometry{margin=0mm}

\title{E. Demaine and Computational Geometry}
\author{Boris A. Zolotov}
\date{\today}

\begin{center}
	\includegraphics[height=28.7cm]{titlepdfpage}
\end{center}

\restoregeometry

\newpage \maketitle \vspace{1cm}

\tableofcontents \vfill \eject

\section{Introduction}

It is very rare that a person is appreciated and recognized for their achievements during the period they are active. Most often it takes time for the humanity to evaluate the significance of certain accomplishments and to pay its creator back with fame.

Sometimes though a prodigy is so great that they cannot remain unnoticed. In this paper we are considering exactly that case: possessing a sheer number of publications and pioneering several topics in mathematics and theoretical computer science, Erik Demaine is a renowned and celebrated scientist who has made some enormous impact in modern computational geometry.

Demaine received bachelor's degree at 14, master's at 15, and Ph.\,D. at 20, which is truly astonishing. As a student he found himself in the department of oceanography, simulating collisions and interactions between tectonic plates. From there he transitioned to computational geometry, since several techniques from this field were used for describing the plates and their movements.

His father had been a visual artist; he had to attend Erik's classes at the university because Erik was so young. When Erik started pursuing computational geometry and algorithms, his father noted many connections between what he had been doing and Erik's studies. This is how he also got into computational geometry, and father and son Demaines have worked on several publications together.

Demaine's research interests range across wide variety of fields. They include algorithms and data structures, games and their computational difficulty, improving web searches, geometry of organic molecules. As any mathematician whose variety of subjects is this wide, Demaine cannot but pursue art in different forms: origami, movie producing, glassblowing.

Demaine's book on Geometric folding algorithms~\cite{DO07} is considered the leading authority when it comes to computational origami, nets, gluing of polyhedra, linkages and rigid constructions.

In this work we are going to cover Erik Demaine's biography and get a glimpse of his scientific results and discoveries that helped shape modern computational geometry. The sources we are using are his recent interview~\cite{yt}, his books and articles.

\section{Chidhood}

When Erik was 7, his father decided it was a good idea to travel around the continent as an alternative educational practice. The reasoning was that experiencing different cultures of the United States by living inside them would help raising the son. The stops of their journey were Miami, Providence, Rhode Island, Chicago, Traverse City (Michigan).

The journey allowed for meeting many people and seeing many places, but most importantly it helped form a bond between the father and the son, since cooperative travel enables those traveling to interact as peers no matter their previous relative status.

It is not though that the family relationships had not already been counting in the interests of both parties: when Erik was even younger, he helped his father design jigsaw puzzles and sell them to toy stores.

Because of the constant change of the location, it was the best solution for Erik to be homeschooled. He advanced quickly in the subjects most easy to him. What was also adding to the efficiency of the studies is that, despite only taking an hour or two each day, they resumed during summer and other holidays. This enabled Erik to not forget what had been studied previously and to not waste time on remembering and repeating it, most critically every fall, when it is done in schools.

Erik soon became interested in computers, he started by learning {\it Basic} and creating a very simplistic game for himself. Since then he has been writing computer programs endlessly~— not games, though.

Since he reached the limits of knowledge of his father, he continued studying maths at an equivalency class at the Halifax University. He was not preparing for any kinfs of exams, still he passed one, which impressed the professor that held the class. Erik proceeded to enroll to the University.

\section{University studies}

Being able to learn quickly at his young age~— he was still 12 at the point~— he “soaked up knowledge” taking the entire set of mathematics and computer science classes. After two years he started doing his research in applied computer science. For his master studies he went to Waterloo, Canada, where he got into theoretical computer science and algorithms.

Demaine continued his graduate studies for six years, since it was his last chance to be a student. He developed his style of research, which is very collaborative and includes working with lots of different people. As a result, he had forty papers and sixty coauthors when the time came to defend. He liked keeping more coauthors than he had papers, and this huge number of collaborators aided him at the MIT interview: MIT valued people who worked well with others. Erik secured his place at MIT as has been working there ever since.

Erik's father followed him during his studies. They occupied a room for family in an University dorm: Erik as a student, and his father as his child. As a result they had more than 20 papers written as a team, and they were offered positions at Universities as a very productive one.

\section{Computational origami}

Pioneered by Demaine (there were no more than a couple publications on the subject before him), the field's concern is to discover the possible advancement of what things can be folded from a piece of paper. Computational origami acquires tools from mathematics and computer science to help process the hardest geometric aspects.

When an approximation of a real-life object is studied, a mathematical model of this object is always needed, which reflects the main properties of it that are of importance for the researcher and omits details and intricacies which do not affect the results much. The mathematical model behind the studies of origami is as follows: given a square sheet of paper, that can not stretch (and therefore has zero Gaussian curvature), one can only fold it in straight lines.

The question arises, what one can really do with only folding. The answer is, surprisingly many things.

The most general problem in the origami setting is given a 3D shape, produce a sequence of folds of a piece of paper that produce approximately that shape. This is where the study of algorithms comes to help, it enables one to translate between shapes and marked points to the construction of sequences of actions.

Another problem is the so-called fold-and-cut problem: given a polygon \(P\) drawn on a piece of paper, can it be made by folding this piece and then making a single straight-line cut. After unfolding the piece back it should be separated into two parts: \(P\) and the complement of \(P\). Certain cases of this problem are known for centuries and have been used as magic tricks entertaining general public.

It turned out that for any polygon \(P\) obtaining it in one cut is possible~\cite{stcut}. This is an example of automating a design challenge, of dedicating an algorithm or a computer program to implement it in an unmanned way.

Departing from origami, the studies of folding have been carried on to cover folding of proteins and robot arms. The point behind this is that we have developed numerous techniques to quickly manufacture flat sheets of various forms, yet manufacturing of 3D parts remains complicated and expensive. Therefore it is tempting to try finding a way to produce a flat shape and then fold it to get a desired 3D object. That would allow for fast making of robots of hihgly customizable shape. Robots could also assemble themselves, this is achieved by introducing motors to a shape that is foldable but still flat.

There are many applications of origami engineering: Space Shuttle manipulators, tools that remove clogging in human arteries, car airbags stored inside the steering wheel. Demaine motivates his interest in origami by it being “cool and art and so on” and by the fact that this field was almost empty and open at the time he arrived to it.

\section{Protein folding}

A protein is an one-dimensional chain of aminoacids that is stored inside a cell in a folded form. The problem with proteins is that we can not observe the folding process and its principles, we can only see some folding results and speculate how it happened and how the protein got into that exact shape.

The problem of reconstructing the sequence of folds giving protein molecules their shape is still unsolved. One can simulate though, if they were nature, how they would design the folding process for the proteins that is efficient and produces a compact enough shape for storage.

The reason for considering proteins in particular is that folding man-made objects such as robot arms is usually complex, therefore the question remains of finding a folding process simple enough to be feasible in nature.

\section{Curved creases and art forms}

Curved creases and how they transform a creased sheet is still a big challenge. One can start looking at it by applying curved creases and watching what happens. Demaine noted that the result is unexpectedly beautiful, so he made several sculptures out of curved creases and submitted them to an art exhibition. This is how he got into art, he has participated in numerous exhibitions ever since.

Demaine suggests looking at any question from two sides: what are the mathematical open problems and possible results are behind it and which art form can you incorporate and use it in. This way one is twice as likely to get a productive output somehow: either as an art form or as a mathematical publication.

Curved creases provide an example of such an approach: only recently has some mathematical progrees been made regarding them, yet they have been used as a sculpture technique for some time already.

\section{Studying foldings}

We mentioned before the book “Geometric Folding Algorithm”~\cite{DO07} written by Erik Demaine that is cited in the first place when it comes to the studies of linkages, origamis and polyhedra. In this section we give a brief review of the core statements it is biult around.

The studies of foldings done by Demaine can be logically divided according to the dimension of the object in question: one-dimensional linkages, two-dimensional pieces of paper, three-dimensional polyhedra. Demaine also divides all the problems related to folding into two groups: design problems~— for one specific shape find if one can fold to that shape, and foldability problems~— find if a class of objects can be folded into a class of shapes.

Design problems can themselves be challenging and somewhat spectacular, they emphasize the geometry of selected pair of objects that are considered. On the other hand, foldabilty problems are more about computational complexity of the problem of folding.

Let us describe the main problems whose answer has been the driving force of development of the whole foldability theory: \begin{enumerate}
	\item[(1D)] Given an algebraic curve in the plane, is there some linkage such that some joint of it traces this curve?
	\item[(2D)] Given a three-dimensional shape, find a sequence of folds and describe the pattern of creases that would allow to create an origami of this shape from a square piece of paper.
	\item[(3D)] Given a convex polyhedron, can its surface be cut along its edges and unfolded to a plane connected not self-intersecting net?
\end{enumerate}

The first problem is very well-known, and the answer for it is positive, even though some technical polishing of the proof was been done until 2002.

As for the second problem, there are no once-and-forever results obtained yet. This is in part why origami even today is an art and not a routine. There is one known algorithm that creates a base for an origami, which is a metric tree reflecting the properties of the desired object; when the base is created, refining the creases is a relatively easy part.

The problem with this algorithm is that it is not proved formally that the shape that is folded using the base that is output by the algorithm does not self-intersect. However, there were many experiments carried out using different instances of the shapes that support that it is indeed the case.

For the third one, the answer is known to be positive if the restriction of cuts being only along edges is removed. So there remains a hope that it is provable that the answer remains positive for more and more general cases.

To cover the issues associated with these three problems a thorough classification of known results, the ones obtained by Demaine himself among them, is done in the book. Different types of linkages, foldings and origamis are listed.

While researching foldings, Demaine pioneered some techniques that have been used extensively in following publications on the subject by other authors, those include, for example, the gluing tree, which is a way to describe a folding of a single polygon (i.\,e.~a net) to a polyhedron.

\section{Conclusion}

Erik Demaine is a scientist with unimaginably great field of view and even greater collective of coauthors, who has done a large number of groundbreaking (or\ldots groundfolding) studies and hopefully has enough time in front of him to perform even more. It would take a long list to cover all the issues looked at and resolved by him, and in this paper we tried to present a short overview of what they are and who is behind them. \newpage

\bibliography{phil}{}
\bibliographystyle{plain}

\end{document}
